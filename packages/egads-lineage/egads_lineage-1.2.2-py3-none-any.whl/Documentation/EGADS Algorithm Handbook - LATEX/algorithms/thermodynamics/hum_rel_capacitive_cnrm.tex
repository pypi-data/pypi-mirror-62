%% $Date: 2012-07-06 17:42:54#$
%% $Revision: 149 $
\index{hum\_rel\_capacitive\_cnrm}
\algdesc{Relative humidity from capacitive probe
}
{ %%%%%% Algorithm name %%%%%%
hum\_rel\_capacitive\_cnrm
}
{ %%%%%% Algorithm summary %%%%%%
Calculates relative humidity using the measured frequency from a capacitive probe.
}
{ %%%%%% Category %%%%%%
Thermodynamics
}
{ %%%%%% Inputs %%%%%%
$Ucapf$ & Vector & Output frequency of the capacitive probe [Hz]\\ 
$T_s$ & Vector & Static temperature [K] \\
$P_s$ & Vector & Static pressure [hPa] \\
$\Delta P$ & Vector & Dynamic pressure [hPa] \\
$C_t$ & Coeff. & Temperature correction coefficient [\%\deg C] \\
$F_{min}$ & Coeff. & Minimal acceptable frequency [Hz] \\
$C_0$ & Coeff. & 0th degree calibration coefficient \\
$C_1$ & Coeff. & 1st degree calibration coefficient \\
$C_2$ & Coeff. & 2nd degree calibration coefficient 
}
{ %%%%%% Outputs %%%%%%
$H_u$ & Vector & Relative humidity [\%]
}
{ %%%%%% Formula %%%%%%
If $Ucapf \leq F_{min}$ then $Ucapf = F_{min}$
%
\begin{displaymath}
H_u = \frac{P_s}{P_s + \Delta P} \left[C_0 + C_1 Ucapf + C_2 Ucapf^2 + C_t (T_s-20)\right] \nonumber
\end{displaymath}
with $T_s$ in $^\circ C$ and $20$ in $^\circ C$.
}
{ %%%%%% Author %%%%%%
CNRM/GMEI/TRAMM
}
{ %%%%%% References %%%%%% 
CAM note on humidity instrument measurements. \cite{Bellec}
}
